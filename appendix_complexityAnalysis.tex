\subsection{Complexity Analysis: Not Feasible}%
\pdfbookmark[2]{Complexity Analysis: Not Feasible}{complexityNotFeasible}%
%
\begin{frame}[t]%
\frametitle{Algorithm Analysis and Comparison}%
\begin{itemize}%
\item \alert{Which of the algorithms is best (for my problem)?}%
\item<2-> Traditional Approach: Complexity Analysis, Theoretical Bounds of Runtime and Solution Quality%
\item<3-> \alert<-9>{Usually not feasible}\uncover<4->{%
\begin{itemize}%
\item analysis extremely complicated\uncover<5->{ since%
\item<5-> algorithms are usually randomized\uncover<6->{ and%
\item<6-> have many parameters (e.g., crossover rate, population size)\uncover<7->{ and%
\item<7-> \inQuotes{sub-algorithms} (e.g., crossover operator, mutation operator, selection algorithm)%
\item<8-> optimization problems also differ in many aspects%
\item<9-> theoretical results only available for toy problems and extremely simplified algorithms\scitep{WWCTL2016GVLSTIOPSOEAP}.%
\item<10-> \inQuotes{performance} has two dimensions (time, result quality), not one\dots%
}}}%
\end{itemize}%
}%
%
\item<11-> \alert{Experimental analysis and comparison is the only practical alternative.}%
%
\end{itemize}%
\end{frame}%